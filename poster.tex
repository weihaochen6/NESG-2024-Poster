% Gemini theme
% https://github.com/anishathalye/gemini
%
% We try to keep this Overleaf template in sync with the canonical source on
% GitHub, but it's recommended that you obtain the template directly from
% GitHub to ensure that you are using the latest version.

\documentclass[final]{beamer}

% ====================
% Packages
% ====================

\usepackage[T1]{fontenc}
\usepackage{lmodern}
\usepackage[size=custom,width=84,height=60,scale=1.0]{beamerposter}
\usetheme{gemini}
\usecolortheme{gemini}
\usepackage{graphicx}
\usepackage{booktabs}
\usepackage{tikz}
\usepackage{pgfplots}
\usepackage{svg}
\pgfplotsset{compat=1.14}

\usepackage[latin9]{inputenc}
\usepackage{geometry}
\geometry{verbose,lmargin=20mm,rmargin=20mm}
\usepackage{amsmath}
\usepackage{amsthm}
\usepackage{multirow}
\usepackage{array}
\usepackage{mathrsfs}
\usepackage{babel}
\usepackage{bm} % 加载bm包来使用\bm命令
\usepackage{enumitem} % 引入宏包以便更灵活地控制列表环境
\usepackage{natbib}
\bibliographystyle{apalike}

\usepackage{lmodern}
\setbeamerfont{title}{size=\small,series=\bfseries,family=\rmfamily} % 减小标题字体大小
\setbeamerfont{normal text}{family=\rmfamily}
\setbeamerfont{block title}{size=\normalsize,series=\bfseries,family=\sffamily}
\setbeamerfont{block body}{size=\small,family=\rmfamily}


% ====================
% Lengths
% ====================

% If you have N columns, choose \sepwidth and \colwidth such that
% (N+1)*\sepwidth + N*\colwidth = \paperwidth
\newlength{\sepwidth}
\newlength{\colwidth}
\setlength{\sepwidth}{0.025\paperwidth}
\setlength{\colwidth}{0.3\paperwidth}

\newcommand{\separatorcolumn}{\begin{column}{\sepwidth}\end{column}}

% ====================
% Title
% ====================
\renewcommand{\baselinestretch}{0.8}
\setlength{\parskip}{0mm}

\title{Adding valid and variance-reducing controls}

\author{ Weihao Chen\inst{1} \and Otilia Boldea\inst{1}  \and Pavel Čížek \inst{1}}

\institute[shortinst]{\inst{1} Department of Econometrics \& OR, Tilburg University }

% ====================
% Footer (optional)
% ====================

\footercontent{
  \href{http://www.nesg.nl/}{http://www.nesg.nl/} \hfill
  NESG Conference 2024, Maastricht \hfill
  \href{mailto:w.chen@uvt.nl}{w.chen@uvt.nl}}
% (can be left out to remove footer)

% ====================
% Logo (optional)
% ====================

% use this to include logos on the left and/or right side of the header:
\logoright{\includegraphics[height=5cm]{departmentlogo.jpg}}
\logoleft{\includegraphics[height=5cm]{Tilburg-University-logo.png}}

% ====================
% Body
% ====================

\begin{document}
\begin{frame}[t]
\begin{columns}[t]
\separatorcolumn

\begin{column}{\colwidth}
  \begin{block}{Introduction}
  \vspace{-5mm}
  \begin{itemize}[label=$\bullet$]
      \item most of the literature has focused on selecting external instruments to increase efficiency.
      \item we focus on selecting controls.
      \item we propose using adaptive group lasso to select valid controls and prove its consistency.
      \item among the valid controls and instruments, we propose a selection criterion to minimize variance of a target parameter estimate and prove its consistency.
       \item applicable to causal inference in micro and macro (local projections).
  \end{itemize}
  \end{block}
  \vspace{-5mm}
\begin{block}{Theory}
\heading{Model, validity condition and types of controls}
Consider the following \textbf{short and long linear model} with parameter of interest $\beta_0$ 
     $$
     y_{t}=x'_{t}\beta_{0}+\varepsilon_{t},\quad y_{t}=x'_{t}\beta_{0}+w'_{t}(c)\gamma_{0}(c)+u_{t}(c)
     $$
     where $w_{t}$ is a $p_w\times 1$  vector of candidate set of controls and $c\in C$ is a selection vector, where $C=\left\{ c=\left(c_{1},\ldots c_{p_{w}}\right)^{\prime};c_{j}=0,1 \text{ for all }j\right\}$. And we also have a finite set of valid instruments, i.e., $p_z\times 1$ vector $z_{t}$. 
     
     \begin{itemize}[label = \bullet]
         \item \textbf{The short moment conditions:}
            $$
    E\left[z_{t}\varepsilon_{t}\right]=E\left[z_{t}(y_{t}-x_{t}^{\prime}\beta_{0})\right]=0
    ,
    $$
         \item \textbf{The long moment conditions:}
         $$
    E\left[z_{t}\left(\varepsilon_{t}-w'_{t}(c)\gamma(c)\right)\right]=0,
    $$
    $$
    E\left[w_{t}(c)\left[\varepsilon_{t}-w_{t}(c)\gamma(c)\right]\right]=0.
    $$
     \end{itemize}
    
    A selection vector $c$  is \textbf{valid} if and only if 
    $$
    E\left[z_{t}w_{t}^{\prime}(c)\right]E\left[w_{t}(c)w_{t}^{\prime}(c)\right]^{-1}E\left[w_{t}(c)\varepsilon_{t}\right] = 0.
    $$
    The above condition holds when either one of following holds:
    \begin{itemize}[label = \bullet]
        \item $E\left[z_{t}w^{\prime}(c)\right]=0$, Instruments Uncorrelated (IU);
        \item $E\left[w_{t}(c)\varepsilon_{t}\right]=0$, Error Uncorrelated (EU).
    \end{itemize}
    The above conditions define four types of  selection vectors: IUEU,IUEC,ICEU,ICEC with first three valid. 

    \heading{Adding EU as moments}
    For the EU types, we notice that it is actually an additional moment condition for $\beta_0$, we show that adding the control as a additional instrument is asymptotically no less efficient compared to adding as a control to short model.

    \textbf{Theorem} For $c_{EU}$ satisfying $E\left[w_{t}(c_{EU})\varepsilon_{t}\right]=0$, under regularity conditions the optimal GMM estimator based on 
    $$
        E(z_t \varepsilon_t) = 0,\quad E\left[w_{t}(c_{EU})\varepsilon_{t}\right]=0
    $$
    is asymptotically no less efficient than the optimal GMM estimator based on long moment conditions with $c_{EU}$.
    \begin{itemize}[label = \triangle]
        \item It is not worse if we add $w_{t}(c_{EU})$ as additional moment and this avoids the problem that pooling EU and IU into short model as controls may destroy validity.
    \end{itemize}

    \heading{Selecting valid controls}
        For $j=1,\cdots,p_w$, we select valid moment conditions $g_j$ of following forms
         \begin{enumerate}[label=(\roman*)]
             \item $E(z_t w_{t,j})=0$;
             \item $E(w_{t,j}\varepsilon_t)=0$;
             \item $E(z_t w_{t,j})E(w_{t,j}\varepsilon_t)=0$.
         \end{enumerate}
    \end{block}
\end{column}

\separatorcolumn

\begin{column}{\colwidth}
    \begin{block}{Theory Continued}
         The estimated selection vectors are  $\hat{c}_g=I(||\hat{a}_{j}||=0)$, where $\hat{a}$ is minimizer of
    $$
    \min_{a}Q(a)=T\sum_{j=1}^{p_{w}}\left(||\hat g_j-a_{j}||^{2}+2\lambda_{g,j}\sqrt{a'_{j}a_{j}}\right),
    $$

    where $\hat g_j$ is the sample analog of the moment condition, $\lambda_g$ are $p_{w}\times 1$ vectors of thresholds with elements $\lambda_{g,j}$ and $a_j$ is the $j$th column of $a$.
    \begin{itemize}[label = \triangle]
        \item When $\varepsilon_t$ is needed, we use residuals $\hat \varepsilon_t$ from short model.
    \end{itemize}
    \textbf{Theorem} With $S_{g}$ the index set for valid moment conditions $g$, $c_g$ the corresponding selection vector, and $S_{g}^{c}=\{1,\cdots,p_{w}\} \setminus S_{g}$, under regularity conditions, if $\sqrt{T}\lambda_{g,j}\xrightarrow{p}\infty$, for $j\in S_{g}$ and $\sqrt{T}\lambda_{g,j}=O_{p}(1)$ for $j\in S_{g}^{c}$, then we have $\hat{c}_g\xrightarrow{a.s.}c_g$.

    \heading{Tuning parameter selection}
    We consider using BIC type moment condition selection criterion to select the tuning parameter $\lambda$
    $$
    \lambda_{g,j} = \frac{\lambda}{||\hat{g}_j||},\quad \text{BIC} = T\sum_{j=1}^{p_{w}}||\hat g_j-\hat a_{j}||^{2} - |\hat c_g|\ln T.
    $$
\vspace{-10mm}
    \heading{Selecting variance-reducing controls}
    
    Following \cite{hall2007information}, we propose using the following selection criterion to select the most efficient specification for the parameter of interest:
    $$ 
    ESC(c)=\ln \left|\hat{V}_{\beta}(c)\right|+\frac{(|c|-1)\ln(T^{1/2})}{T^{1/2}},
    $$
    where $\hat{V}_{\beta}(c)$ is a consistent estimator of the asymptotic variance of $\beta$.
    
    \textbf{Theorem} Under regularity conditions, if most parsimonious set of valid and efficiency-increasing controls is $ c_e$ with $\hat c_e$ minimizing $ESC(c)$, then $\hat c_e \xrightarrow{\mathrm{a.s.}} c_e$. 
    \begin{itemize}[label = \bullet]
        \item \textbf{Special case 1:} $w_t$ are all IU, as usual in the local projection literature, no selection for validity needed.
         \item \textbf{Special case 2:} $w_t$ are all EU serving as valid IV thus no selection for validity is needed and the procedure is reduced to \cite{hall2007information}.
    \end{itemize}
  \end{block}
  
  
  \begin{block}{Simulation}
  % Begin of the content to be in parallel
    \begin{minipage}[c]{0.5\textwidth}
    \textbf{DGP:}
     $$
  y_{t}=\beta_{0}x_{t}+u_{t}, 
\vspace{-3mm}
  $$  
  $$ 
  x_{t}= z_{t}\gamma_z +  w_{t,1}\gamma_{1}+w_{t,2}\gamma_{2}+v_{t}
  \vspace{-3mm}
   $$
    $$
  u_{t}=w_{t,3}\gamma_{3}+w_{t,4}\cdot 0+w_{t,5}\cdot 0+e_{t}
  \vspace{-3mm}
  $$
  $$
    w_{t,6}=\pi_1 u_t + \pi_2w_{t,1} +\eta_{t,1}
  \vspace{-3mm}
  $$
  $$
    w_{t,7}=\theta_1 u_t + \theta_2z_{t} +\eta_{t,2}
  \vspace{-3mm}
  $$
  \end{minipage}
  \hfill 
  \begin{minipage}[c]{0.50\textwidth}
        \begin{itemize}[label = \bullet]
            \item adapted from \cite{cheng2015select};
            \item $w_{t,4}$,$w_{t,5}$ are redundant;
            \item $E(w_{t,1}z_t)=0$,$E(w_{t,2}z_t)=0$;
            \item iid errors, allow for heteroskedasiticity and autocorrelated errors.
            \item $\pi_1$ small compared to $\pi_2$ to make $w_{t,6}$ variance-increasing.
        \end{itemize}
        
  \end{minipage}

\begin{minipage}[c]{0.3\textwidth}
 \begin{table}
\caption{IU and EU types}
\renewcommand{\arraystretch}{1.5}
\begin{centering}
\begin{tabular}{ccc}
\hline 
 & IU & IC\tabularnewline
\hline 
EU & $w_{t,1}$,$w_{t,4}$,$w_{t,5}$ & $w_{t,2}$\tabularnewline
EC & $w_{t,3}$,$w_{t,6}$ & $w_{t,7}$\tabularnewline
\hline 
\end{tabular}
\par\end{centering}
\centering{}
\end{table}
\end{minipage}
  \hfill
\begin{minipage}[c]{0.7\textwidth}
 \begin{table}
   \renewcommand{\arraystretch}{1.5}
\begin{centering}
\caption{Control types}
\begin{tabular}{c|c|c|c}
\hline 
\multicolumn{3}{c|}{type} & control\tabularnewline
\hline 
\multirow{4}{*}{valid} & \multirow{2}{*}{variance-reducing} & \textbf{as IV} & \bm{$w_{t,1}$,$w_{t,2}$}\tabularnewline
\cline{3-4} \cline{4-4} 
 &  & \textbf{as control} & \bm{$w_{t,3}$}\tabularnewline
\cline{2-4} \cline{3-4} \cline{4-4} 
 & \multicolumn{2}{c|}{redundant} & $w_{t,4}$,$w_{t,5}$\tabularnewline
\cline{2-4} \cline{3-4} \cline{4-4} 
 & \multicolumn{2}{c|}{variance-increasing} & $w_{t,6}$\tabularnewline
\hline 
\multicolumn{3}{c|}{invalid} & $w_{t,7}$\tabularnewline
\hline 
\end{tabular}
\par\end{centering}
\centering{}
\end{table}
\end{minipage}
  
 
  \end{block}
  
  \end{column}
  
\separatorcolumn

\begin{column}{\colwidth}
 \begin{block}{Simulation Continued}  
 \usetikzlibrary{shapes.geometric, arrows.meta, positioning}

% 定义样式
\tikzset{
    start/.style={
        draw, rectangle, rounded corners, minimum width=2.5cm, minimum height=1cm, align=center, fill=green!50, text=black, font=\bfseries
    },
     input/.style={ 
        draw, trapezium, trapezium left angle=70, trapezium right angle=110, minimum width=2cm, minimum height=1cm, align=center, fill=purple!20, text=black, text width=5cm, font=\bfseries
    },
    stop/.style={
        draw, rectangle, rounded corners, minimum width=2.5cm, minimum height=1cm, align=center, fill=red!50, text=black, font=\bfseries
    },
    process/.style={
        draw, rectangle, minimum width=3cm, minimum height=1cm, align=center, fill=blue!20, text=black, text width=6cm, font=\bfseries
    },
    estimate_process/.style={
        draw, rectangle, minimum width=3cm, minimum height=1cm, align=center, fill=blue!20, text=black, text width=8cm, font=\bfseries
    },
    add_control_process/.style={
        draw, rectangle, minimum width=3cm, minimum height=1cm, align=center, fill=blue!20, text=black, text width=8cm, font=\bfseries
    },
    select_process/.style={
        draw, rectangle, minimum width=3cm, minimum height=1cm, align=center, fill=blue!20, text=black, text width=4cm, font=\bfseries
    },
    categorize_process/.style={
        draw, rectangle, minimum width=3cm, minimum height=1cm, align=center, fill=blue!20, text=black, text width=5cm, font=\bfseries
    },
    decision/.style={
        draw, diamond, aspect=2, minimum width=3cm, minimum height=1cm, align=center, fill=yellow!50, text=black,text width=2cm,font=\bfseries
    },
    arrow/.style={
        thick,->,>=stealth
    },
    every node/.style={font=\rmfamily}
}
 \vspace{-5mm}
 \textbf{Implementation algorithm}
 \vspace{5mm}
 \\
\begin{tikzpicture}[node distance=2cm and 1.5cm, auto, scale=0.8, transform shape]  % scale the entire drawing

    % 节点
    \node (start) [start] {Start};
    \node (input1) [input, below=of start] {Data input: \bm{$(y_t,x_t,z_t,w_t)$}.};
    \node (validity) [decision, below=of input1] {Known \\IU, EU types?};  % 判断节点
    \node (estimate_short) [process, right=of validity] {Estimate short model; \\get residuals.};
    \node (Add_controls) [add_control_process, below=of validity] {Add IUEU,ICEU as moments;\\ add ECIU as control;\\throw out ECIC.};
    \node (estimate) [estimate_process, right=of Add_controls] {Estimate \bm{$2^{|\text{IUEU}|+|\text{ICEU}|}\times 2^{|\text{IUEC}|}$} specifications;\\ Calculate ESC.} ;
    \node (select) [select_process, right=of estimate_short] {Select \\IU(IC),\\EU(EC).};
    \node (categorize) [categorize_process, right=of select] {Categorize: \\IUEU,IUEC,\\ICEU,ICEC.};  % Adjusted position
    \node (find) [process, right=of estimate] {Find ESC minimized specification.};
    \node (stop) [stop, right=of find] {Stop};

    % 箭头
    \draw [arrow] (start) -- (input1);
    \draw [arrow] (input1) -- (validity);
    \draw [arrow] (validity) -- node[above] {No} (estimate_short);
    \draw [arrow] (estimate_short) -- (select)
    \draw [arrow] (select) -- (categorize);
    \draw [arrow] (validity) -- node[left] {Yes} (Add_controls);
    % Improved path with straight and smooth turns
    \draw [arrow] (categorize.south) -- ++(0,-2) -| (Add_controls);
    \draw [arrow] (Add_controls) -- (estimate);
    \draw [arrow] (estimate) -- (find);
    \draw [arrow] (find) -- (stop);

\end{tikzpicture}
        \begin{table}
        \renewcommand{\arraystretch}{1.5}
        \renewcommand*\familydefault{\sfdefault}
        \caption{Selection probabilities}
         \vspace{-3mm}
        \begin{centering}
        \begin{tabular}{cccccc}
        \hline 
        Sample size & $P(\text{IU})$ & $P(\text{EU})$ & $P(\text{IU\&EU})$ & $P(\text{Efficient|IU\&EU})$ & $P(\text{Efficient})$\tabularnewline
        \hline 
        $T=500$ & 0.8220 & 0.8870 & 0.7360 & 0.9850 & 0.8720\tabularnewline
        $T=1000$ & 0.9810 & 0.9580 & 0.9390 & 0.9970 & 0.9530\tabularnewline
        $T=2000$ & 0.9990 & 0.9930 & 0.9920 & 0.9980 & 0.9900\tabularnewline
        \hline 
        \end{tabular}
        \par\end{centering}
        \centering{}
        \end{table}
     \vspace{-5mm}
    \begin{table}
    \renewcommand{\arraystretch}{1.5}
    \renewcommand*\familydefault{\sfdefault}
     \begin{centering}
     \caption{Estimation results comparison}
     \vspace{-3mm}
    \begin{tabular}{ccccccc}
    \hline 
     & \multicolumn{3}{c}{$T=500$} & \multicolumn{3}{c}{$T=1000$}\tabularnewline
    \hline 
     & Bias & SD & RMSE & Bias & SD & RMSE\tabularnewline
    Selected | IU\&EU & 0.0008 & 0.0412 & 0.0411 & 0.0028 & 0.0266 & 0.0268\tabularnewline
    \textbf{Selected} & \textbf{-0.0019} & \textbf{0.0509} & \textbf{0.0509} & \textbf{0.0013} & \textbf{0.0314} & \textbf{0.0314}\tabularnewline
    Short & -0.0018 & 0.0991 & 0.0990 & -0.0003 & 0.0683 & 0.0683\tabularnewline
    Oracle & 0.0003 & 0.0407 & 0.0407 & 0.0027 & 0.0266 & 0.0267\tabularnewline
    \hline 
    \end{tabular}
    \par\end{centering}
    \centering{}
    \end{table}

 \end{block}


    \begin{block}{Conclusions and future research}
    \vspace{-2mm}
         \begin{itemize}[label = \hookrightarrow]
            \item We consider adding controls to a short model to increase efficiency of a target parameter.
            \item The proposed procedure effectively selects valid and variance-reducing controls.
            \item Extension to high dimension for special cases.
            \item Extension to weak identification.
            \item Applications for local projection.
        \end{itemize}
    \end{block}

\begin{block}{References}
\bibliography{poster}
\end{block}

\end{column}

\separatorcolumn
\end{columns}
\end{frame}

\end{document}
